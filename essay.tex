\documentclass[a4paper,12pt]{article}
\usepackage{partiiiessay}

\title{Serre's conjecture\vspace{-11pt}} %TODO subtitle?
\author{Alex J. Best}

% TODO List:
% Can any refs be made more specific?
% sort out serre75 reference, what is volume/number?
% use semicolons / generally check consistency of notation

\begin{document}
\maketitle
\vspace{-50pt}
\tableofcontents
\clearpage


% ************************************************************************
% Introduction
% ************************************************************************
\section{Introduction}
In 1987 Jean-Pierre Serre published a paper \cite{Serre87}, ``Sur les repr\'esentations modulaires de degr\'e 2 de $\Gal(\Qb/\QQ)$'', in the Duke Mathematical Journal.
In this paper Serre outlined a conjecture detailing a precise relationship between certain mod $p$ Galois representations and specific mod $p$ modular forms.
This conjecture and its variants have become known as Serre's conjecture, or sometimes \emph{Serre's modularity conjecture} to distinguish it from other conjectures Serre has made.
The conjecture has since been proven correct by the work of numerous people, culminating with that of Khare, Wintenberger and Kisin, published in 2009 \cite{KWI,KWII,Kisin}.

Here we provide a motivated account of the standard form of the conjecture before going on to compute some explicit examples and examining some interesting consequences.

Beyond the original paper there are many very good accounts of Serre's statement, including Cais \cite{Cais}, Edixhoven \cite{Edixhoven} (both of which use Katz's definition of mod $p$ modular forms), and Darmon \cite{Darmon} (which stays closer to the original article).
There is also a chapter by Ribet--Stein \cite{RibetStein}. %TODO separate more expositary, move researchy onese to relevant sections.
Alex Ghitza has prepared a translation of part of Serre's paper \cite{Ghitza} which has been helpful.
%I consulted these articles while preparing the current essay and they were of great help. TODO


% ************************************************************************
% Background
% ************************************************************************
\section{Background}
Here we fix several definitions and key results that will be relevant when discussing Serre's conjecture.

%TODO: mod p forms, of specific weight level char
\subsection{Modular forms}
We assume material relating to classical modular forms, and here only look at the passage to \emph{mod $p$ modular forms} as these are a key part of Serre's conjecture and as there is some amount of choice in how these forms are defined.

\begin{defn} %TODO also non cuspidal??
Let $S_k(N,\,\varepsilon;\,\QQ)$ be the space of cusp forms of level $k$, weight $N$ and character $\varepsilon$ whose $q$-expansion coefficients lie in $\QQ$.

Given a mod $p$ character
\[
\varepsilon\colon (\ZZ/N\ZZ)^* \to \Fb_p^*
\]
we may lift to a character
\[
\hat{\varepsilon}\colon (\ZZ/N\ZZ)^* \to \Zb^*. %TODO
\]

We let the space of \emph{cuspidal mod $p$ modular forms} of weight $k$, level $N$ and character $\varepsilon\colon (\ZZ/N\ZZ)^* \to \Fb_p^*$ be the subspace of $\Fb_p[[q]]$ obtained by reducing mod $p$ the $q$-expansions of forms in $S_k(N,\,\hat{\varepsilon};\,\QQ)$.
We denote this space by
\[
S_k(N,\,\varepsilon;\,\Fb_p).
\]
\end{defn}

\begin{defn}
As for standard modular forms, we say a mod $p$ cusp form
\[
f = \sum_{n\ge 1} a_n q^n,\,(a_n\in\Fb_p)
\]
is \emph{normalised} if $a_1 = 1$.
\end{defn}

\begin{prop}\label{prop:pm1}
If $f$ and $g$ are two mod $p$ modular forms of weights $k$ and $k'$ respectively, whose $q$-expansions are equal, then
\[
k \equiv k' \pmod{p-1}.
\]
\end{prop}
\begin{proof}
\end{proof}

\begin{ex}
\end{ex}

There is an important operator on the space of mod $p$ modular forms which we shall study more in \cref{sec:galassoc}.

\begin{defn}\label{def:theta}
The $\Theta$ operator is defined by its action on $q$-expansions by
\[
\Theta\left(\sum_{n\ge 0} a_nq^n\right) = q\ddq \left(\sum_{n\ge 0} a_nq^n\right) = \sum_{n\ge 0} na_nq^n.
\]
\end{defn}

It is clear from the $q$-expansions that $\Theta$ maps arbitrary modular forms to cusp forms, and also that $\Theta$ preserves the set of normalised forms.

\begin{prop}
If $f\in S_k(N,\,\epsilon;\,\Fb_p)$ then
\[
\Theta(f)\in S_{k+p+1}(N,\,\epsilon;\,\Fb_p).
\]
\end{prop}
\begin{proof}
\end{proof}

\begin{prop}
$\Theta$ preserves eigenforms.
\end{prop}
\begin{proof}
\end{proof}


\subsection{Galois representations}\label{sec:gals}
%TODO refs, wiese?
\begin{defn}
An \emph{$n$-dimensional mod $p$ Galois representation} is a homomorphism
\[
\rho\colon \GQ \to \GL_n(\Fb_p).
\]
Similarly, an \emph{$n$-dimensional $p$-adic Galois representation} is a homomorphism
\[
\rho\colon \GQ \to \GL_n(\QQ_p).
\]

Unless otherwise stated a Galois representation will always refer to a mod~$p$ Galois representation.
\end{defn}

Here we deal mostly with 1 and 2 dimensional mod $p$ Galois representations.
Those of dimension 1 (i.e. maps $\phi\colon\GQ \to \Fb_p^*$) are called \emph{characters}.

Recall that $\GQ$ is defined as the inverse limit of $\Gal(K/\QQ)$ as $K$ ranges over all number fields.
So the group $\GQ$ naturally has the profinite topology, where the open subgroups are the subgroups of finite index.
Our mod $p$ representations are always continuous with respect to this topology and the discrete topology on $\GL_n(\Fb_p)$, though we will often still say as such to remind ourselves.

\begin{rmk}\label{rmk:ctsfin}
The continuity condition for mod $p$ Galois representations reduces to having an open kernel and so continuous mod $p$ Galois representations always have finite image.
\end{rmk}

Although our main object of study is $\GQ$ it will be very useful for us to take a prime $\ell$ and also consider representations of
\[
\Gl = \Gal(\Qb_\ell/\QQ_\ell).
\]
Indeed such representations can be obtained from those of $\GQ$ using an inclusion
\[
\Qb \hookrightarrow \Qlb
\]
to define a restriction map
\[
\Gl \to \GQ.
\]
In fact due to Krasner's lemma \cite[p. 238]{Cohen} the map $\Gl \to \GQ$ is injective and so we may view $\Gl$ as a subgroup of $\GQ$.
The way this subgroup sits inside $\GQ$ depends on the choice of embedding $\Qb \hookrightarrow \Qlb$ and varies by conjugation as this embedding changes.

The group $\Gl$ has several important subquotients which will be helpful for us to study restrictions of representations to.
\begin{defn}\label{def:inert}
The first is the \emph{inertia subgroup} $I_\ell$ which is defined as the kernel of the map
\[
\Gl \twoheadrightarrow \Gal(\Fb_\ell/\FF_\ell)
\]
obtained by quotienting out $\Qlb$ by its maximal ideal.
The group $\Gal(\Fb_\ell/\FF_\ell)$ is topologically cyclic, generated by the Frobenius morphism $x \mapsto x^\ell$.
We can then let $\Frob_\ell\in \Gl$ be a preimage of this morphism under the restriction map, this is only defined up to conjugation.


Next the \emph{wild} inertia group $I_{\ell,w}$ is the maximal pro-$\ell$-subgroup of $I_\ell$ and the \emph{tame} inertia group is the quotient
\[
I_{\ell,t} = I_\ell / I_{\ell,w}.
\]

Finally we may define a series of subgroups of $\Gl$ that study the higher ramification.
Let $\nu_l$ be the extension of the $\ell$-adic valuation to $\Qlb$ and define
\[
\G_{\ell,u} = \{\sigma \in \Gl : \nu_l(\sigma(x) - x) \ge u + 1\ \forall x \in \cO_{\Qlb} \}.
\]
The $\G_{\ell,u}$ form a descending chain as $u$ ranges over the integers that includes several groups we have already mentioned
\[
\Gl = \G_{\ell,-1} \supseteq I_\ell = \G_{\ell,0} \supseteq  I_{\ell,w} = \G_{\ell,1} \supseteq\G_{\ell, 2}\supseteq\cdots.
\]
\end{defn}

\begin{defn}
We say a Galois representation $\rho$ is \emph{unramified} at $\ell$ if $\rho|_{I_\ell}$ is trivial.
Otherwise, we say $\rho$ is \emph{ramified} at $\ell$.

Similarly we say $\rho$ is \emph{tamely ramified} at $\ell$ if $\rho|_{I_{\ell,w}}$ is trivial.
\end{defn}

\begin{defn}
Let $\phi\colon\GQ \to K^*$ be a character for some field $K$ and fix an embedding $\Qb\hookrightarrow \CC$.
We may then view complex conjugation as an element $c\in \GQ$, looking at its image $\phi(c)$ we see it is an element of order 2 in $K^*$, so $\phi(c)$ must be $\pm 1$.
If $\phi(c) = -1$ we say $\phi$ is \emph{odd}, otherwise we say $\phi$ is \emph{even} (though we shall mostly be concerned with distinguishing odd representations here).%TODO

Now given any Galois representation
\[
\rho\colon\GQ \to \GL_n(K),
\]
we define the parity of $\rho$ to be that of the character $\det\rho$.
\end{defn}

\begin{defn}
Each character
\[
\phi \colon \GQ \to \Fb_p^*
\]
has finite image and so factors through some $\FF_{p^n}$, the smallest $n$ for which this can happen is called the \emph{level} of the character.
\end{defn}


The Kronecker--Weber theorem tells us that any character $\rho\colon\GQ \to \Fb_p^*$ factors as %TODO
\[
\rho\colon\GQ \to \Gal(\QQ(\zeta_n)/\QQ)\xrightarrow{\phi_n}\Fb_p^*,
\]
where $\zeta_n$ is a primitive $n$th root of unity.
Thus characters of the Galois group are in bijection with Dirichlet characters
\[
(\ZZ/N\ZZ)^* \xrightarrow{\phi_n}\Fb_p^*.
\]



\begin{defn}
The restriction map
\[
\GQ \to \Gal(\QQ(\zeta_p)/\QQ)
\]
gives an action of $\GQ$ on the primitive $p$th roots of unity, where each $\sigma\in\GQ$ sends
\[
\zeta_p\mapsto \zeta_p^{n_\sigma},
\]
with $n_\sigma \in (\ZZ/p\ZZ)^*$.
So letting $\chi_p(\sigma) = n_\sigma$ gives a character
\[
\chi_p\colon\GQ \to\FF_p^*\hookrightarrow\Fb_p^*,
\]
called the mod $p$ cyclotomic character.
By definition this character is of level 1.
\end{defn}

\begin{rmk}
We note some important properties of the cyclotomic character.

First, if $\ell \ne p$ then
\[
\chi_p(\Frob_\ell) = \ell,
\]
as.
Also fixing an embedding $\Qb\hookrightarrow\CC$ and considering complex conjugation $c\in\GQ$ we see that it takes $\zeta_p \mapsto\zeta_p^{-1}$ and hence
\[
\chi_p(c) = -1.
\]
\end{rmk}

For each $n \ge 1$ we now distinguish $n$ special characters of $I_{p,t}$ of level $n$ which will allow us to describe all characters of a particular level.

\begin{defn}\label{def:fund}
Recall that $I_{p,t}$ can be identified with
\[
\lim_{\longleftarrow} \FF_{p^n}^*
\]
and so we have a natural map
\[
\psi\colon I_{p,t} \to \FF_{p^n}^*
\]
for each $n$.
The \emph{fundamental characters} of level $n$ are defined by extending $\psi$ to an $\Fb_p$-character via the $n$ embeddings $\FF_{p^n}^* \hookrightarrow \Fb_p^*$. %TODO

While any individual fundamental character is not canonical, the set of all of them of a particular level is canonical. %TODO
\end{defn}

\begin{rmk}\label{rmk:prodchar}
The embeddings are all obtained from any chosen one by applying Frobenius and as such the product of all fundamental characters of level $n$ is the same as the composition of the norm map $\FF_{p^n}^* \to \FF_p^*$ with any one.
So this product will always be the unique fundamental character of level 1.
\end{rmk}


%TODO concequences of fund chars
%TODO cyclotomic


% ************************************************************************
% Obtaining Galois representations from modular forms
% ************************************************************************
\section{Obtaining Galois representations from modular forms}
The two concepts just introduced, modular forms and Galois representation, appear at first glance not to be particularly related to each other.
However in reality they are inextricably linked, and exploring the links between them will be the goal of the rest of this essay.

We will start with a historically important example that provides the first glimpse of the behaviour we will be looking at.

\begin{ex}\label{ex:delt}
Let
\[
\Delta = \sum_{n \ge 1} \tau(n) q^n
\]
be the unique normalised cusp form of weight 12 for $\Gamma_1(1) = \SL_2(\ZZ)$.
The coefficients of this $q$-expansion were studied by Ramanujan and $\tau(n)$ became known as the Ramanujan $\tau$ function.
The properties of this function provide the first glimpse of behaviours that extend to more general systems of Hecke eigenvalues.

Various people, including Ramanujan himself (in the mod 691 case) had found congruences involving the coefficients $\tau(n)$.
Below are a few examples, though more exist for higher powers of these primes.
\begin{align}\label{eq:taucong}
\tau(p) &\equiv 1 + p^{11} \pmod{2^8}\text{, if } p \ne 2,\\
\tau(p) &\equiv p^2 + p^9 \pmod{3^3}\text{, if } p \ne 3,\\
\tau(p) &\equiv p + p^{10} \pmod{5^2},\\
\tau(p) &\equiv p + p^4 \pmod{7},\\
\tau(p) &\equiv\left.\begin{cases}
0\pmod{23} & \text{ if } \left(\frac{p}{23}\right) = -1,\\
2\pmod{23} & \text{ if is of the form } p = u^2 + 23v^2,\\
-1\pmod{23} & \text{ otherwise},\\
\end{cases}\right\}\text{ if } p \ne 23,\\
\tau(p) &\equiv 1 + p^{11} \pmod{691}.
\end{align}

In order to explain these congruences Serre predicted \cite{Serre67} the existence of $p$-adic Galois representations $\rho_p$ for each prime $p$ such that
\begin{enumerate}
\item $\tr(\rho_p (\Frob_\ell)) = \tau(\ell)$ for all $\ell \ne p$,
\item $\det(\rho_p (\Frob_\ell)) = \ell^{11}$ for all $\ell \ne p$.
\end{enumerate}

The congruences would then follow from these Galois representations being of a specific form 

These representations were constructed shortly after by Pierre Deligne \cite{Deligne}.
In doing so he reduced Ramanujan's conjecture that $|\tau(p)| \le 2p^{11/2}$ to the Weil conjectures.

While we do not 
\end{ex}
%TODO example of ec ? then more general form -> gal rep

%TODO attaching gal reps to forms
Given the above example it is natural to wonder whether such a relationship holds more generally.
Indeed Serre also asked if one could associate to each normalised cuspidal eigenform a Galois representation whose traces of Frobenius elements match the $q$-expansion coefficients mod $p$.
Serre's conjectures on this led to the following: %TODO ref

\begin{thm}\label{thm:assoc}
Let $k \ge 2$, $N \ge 1$ and $\varepsilon\colon (\ZZ/N\ZZ)^* \to \Fb_p^*$. Given a normalised cuspidal $f\in S_k(N,\,\varepsilon;\,\Fb_p)$ there exists a two-dimensional mod $p$ Galois representation $\rho_f$ such that
\begin{enumerate}[(i)]
\item $\rho_f$ is semi-simple,
\item $\rho_f$ is unramified outside $Np$,
\item $\tr(\rho_f (\Frob_\ell)) = a_\ell$ for all $\ell \nmid Np$,
\item $\det(\rho_f (\Frob_\ell)) = \varepsilon(\ell)\ell^{k-1}$ for all $\ell \nmid Np$. %TODO factor a la diamond-im
\end{enumerate}
We often refer to the representation $\rho_f$ as arising from, or being attached to, $f$.
\end{thm}

The construction of these representations is due to Shimura when $k = 2$ \cite{Shimura}, and Deligne when $k > 2$ \cite{Deligne}.
There is also a similar statement for $k = 1$ due to Deligne--Serre \cite{DeligneSerre}.
(There is an English translation of Deligne's paper available from the IAS \cite{DeligneEng}, it has nicer typesetting too.) %TODO check old

In fact the representations obtained by these authors are $p$-adic Galois representations $\rho_f\colon \GQ \to \GL_2(\QQ_p)$ and the representations in the theorem are then obtained from the $p$-adic one by reducing and semisimplifying (see \cref{def:semisimp}). %TODO
The mod $p$ representations are the ones that we will be most interested in from here on however.

\begin{rmk}\label{rmk:detrho}
Looking at the representation $\rho_f$ coming from this theorem we can see that as
\[
\det(\rho_f(\Frob_\ell)) = \chi_p^{k-1}(\Frob_\ell)\varepsilon(\Frob_\ell)
\]
for all $\ell \nmid Np$ (here viewing $\varepsilon$ as character of $\GQ$ now) and by applying Chebatorev we get %TODO
\[
\det\rho_f = \varepsilon\chi_p^{k-1}.
\]
Hence as $\varepsilon(c)f = \langle -1 \rangle f = (-1)^k f$ (by looking at the action of $-\Id$ on $f$) we calculate %TODO -I?
\[
\varepsilon(c)\chi_p^{k-1}(c) = (-1)^k(-1)^{k-1} = -1
\]
and so $\det\rho_f$ must be odd.
\end{rmk}

%TODO when

We will look at some more properties of this construction in \cref{sec:galassoc}, but now we move on to the main conjecture.


% ************************************************************************
% Serre's Conjecture
% ************************************************************************
\section{Serre's conjecture}
\subsection{The qualitative form}
Given the above result it is natural to ask about the converse statement, given a Galois representation satisfying some necessary conditions, does it arise from an eigenform?
Serre's conjecture is that the answer to this question is yes, all Galois representations that could possibly arise from an eigenform based on the remarks following \cref{thm:assoc} do so.

The conjecture naturally comes into two parts, one weaker existence statement, and another refined form that makes exact predictions about the quantities involved.
We look at the existence statement, or \emph{qualitative form} first.

\begin{conjecture}[Serre's conjecture, qualitative form]\label{conj:qual}
Let $\rho\colon \GQ \to \GL_2(\Fb_p)$ be a continuous, odd, irreducible Galois representation.
Then there exists a normalised cuspidal mod $p$ eigenform $f$ such that $\rho$ is isomorphic to $\rho_f$, the Galois representation associated to $f$.
\end{conjecture}

This is already a very useful thing to know: any statement one could prove about Galois representations attached to  modular forms, by using the theory of these forms for example, would hold for all odd 2-dimensional Galois representations (see \cref{sec:tan,sec:artin} for examples of this). %TODO irred?
%TODO elaborate on importance / use

This conjecture (at least for Galois representations unramified outside $p$) appeared much earlier than the Duke paper and is mentioned by Serre in 1975 \cite[sec. 3]{Serre75}.


\subsection{The refined form}
Given the above statement it is natural to ask about the properties of the form $f$ whose existence is claimed. %TODO natural
Can anything be said about the weight and level of $f$ based only on the properties of $\rho$?
Serre also conjectured that the answer to this question is yes.
He defined a weight, level and character for each $\rho$ such that there should be a form $f$ of that weight, level and character that $\rho$ is attached to.
In a slightly backwards manner we will first state this refined form of the conjecture, before moving on to motivate and define the quantities $N(\rho)$, $k(\rho)$ and character
\[
\varepsilon(\rho)\colon (\ZZ/N(\rho)\ZZ)^* \to \Fb_p^*
\]
used in the statement.

\begin{conjecture}[Serre's conjecture, refined form]\label{conj:ref}
Let $\rho\colon \GQ \to \GL_2(\Fb_p)$ be a continuous, odd, irreducible Galois representation.
Then there exists a normalised eigenform
\[
f\in S_{k(\rho)}(N(\rho),\,\varepsilon(\rho),\,\Fb_p)
\]
whose associated Galois representation $\rho_f$ is isomorphic to $\rho$.

Moreover the $N(\rho)$ and $k(\rho)$ are the minimal weight and level for which there exists such a form $f$.
\end{conjecture}

This conjecture is very bold, even given the existence statement of \cref{conj:qual} it is not clear that a minimal weight and level should exist simultaneously, let alone be given by the relatively straightforward (though intricate) description we are about to detail.

From now on we refer to a Galois representation $\rho$ satisfying the hypotheses of this conjecture as being of \emph{Serre type}.
%We also call a normalised cuspidal eigenform of weight $k$, level $N$ and character $\varepsilon$ a form of type $(k,\,N,\,\varepsilon)$. %TODO at end?


\subsection{Results on Galois representations associated to modular forms}\label{sec:galassoc}
In order to try and understand which forms can give rise to a particular representation it is useful to take an arbitrary form and study the properties of the representation attached to it, in an attempt to see what information about the form may be recovered.
This has been done by several people and the information which will be important to us is contained in the following theorems. %TODO

We fix a prime $p$ and a normalised eigenform $f \in S_k(N,\,\varepsilon;\,\Fb_p)$ with $q$-expansion
\[
f = \sum_{n\ge 1} a_nq^n.
\]
Let $\rho_f$ be the mod $p$ Galois representation attached to $f$ by \cref{thm:assoc}.
Concerning the conductor of $\rho_f$ we have the following result due to Carayol and Livn\'e \cite{Carayol, Livne}.

\begin{thm}\label{thm:level}
Let $N(\rho_f)$ be the level associated to $\rho_f$ (which we will define explicitly shortly), then
\[
N(\rho_f)|N.
\]
\end{thm}

Given this it is natural to hope that any Galois representation of Serre type arises from a form of level exactly $N(\rho)$, of course we still have yet to define this quantity!

We can also make interesting observations concerning the restriction of $\rho_f$ to $\Gp$, for this there are two theorems depending on whether $a_p \ne 0$ (the \emph{ordinary} case) or otherwise (the \emph{supersingular} case).

\begin{thm}[Deligne]\label{thm:ordinary}
Suppose $2\le k\le p+1$ and $a_p \ne 0$ then $\rho_{f,p}|_{\Gp}$ is reducible, moreover, letting $\lambda(a)\colon \Gp \to \Fb_p^*$ be the unramified character of $\Gp$ that takes each $\Frob_p \in \Gp /I_p$ to $a\in\Fb_p^*$, we have
\[
\rho_{f,p}|_{\Gp} = \begin{pmatrix} \chi_p^{k-1}\lambda(\varepsilon(p)/a_p) & * \\ 0 & \lambda(a_p)\end{pmatrix},
\]
up to conjugation in $\GL_2(\Fb_p)$.

In particular when we look at the restriction to inertia we get
\[
\rho_{f,p}|_{I_{p}} = \begin{pmatrix} \chi_p^{k-1} & * \\ 0 & 1\end{pmatrix}.
\]
\end{thm}

A proof of this result when $k \le p$ is given in \cite{Gross} and the general case was originally proved in an unpublished letter from Deligne to Serre.
%TODO add in others, wiles?

\begin{thm}[Fontaine]\label{thm:super}
Suppose $2\le k \le p +1$ and $a_p = 0$ then $\rho_{f,p}|_{\Gp}$ is irreducible, moreover, letting $\psi_1$ and $\psi_2$ be the two fundamental characters of level 2, we have
\[
\rho_{f,p}|_{I_p} = \begin{pmatrix} \psi_1^{k-1} & 0 \\ 0 & \psi_2^{k-1}\end{pmatrix}
\]
up to conjugation in $\GL_2(\Fb_p)$.
\end{thm}

Finally we observe how the $\Theta$ operator changes the representation obtained.

\begin{prop}\label{prop:theta}
Let
\[
\Theta\colon S_k(N,\,\epsilon;\,\Fb_p)\to S_{k+p+1}(N,\,\epsilon;\,\Fb_p)
\]
be the operator defined in \cref{def:theta}.
Then if $f \in S_k(N,\,\epsilon;\,\Fb_p)$ is a normalised eigenform the Galois representation associated to $\Theta(f)$ is
\[
\rho_{\Theta(f)} \cong \chi_p\otimes\rho_{f}.
\]
\end{prop}
\begin{proof}
Look at the $q$-expansions. %TODO
\end{proof}

So applying $\Theta$ bumps the level up by $p+1$ and twists the associated representation by $\chi_p$.

It is worth noting that proofs of some of these theorems came after Serre's paper.
However it seems likely that observations of the above results in specific examples informed the recipe below.


\subsection{The optimal level}\label{subsec:level}
Assume that we have a Galois representation $\rho\colon \GQ \to \GL_2(\Fb_p)$ of Serre type.
We now define the integer $N(\rho) \ge 1$ which plays the role of the optimal level in the refined conjecture.

We can equivalently view our representation $\rho$ as a homomorphism
\[
\GQ \to \Aut(V),
\]
where $V$ is a two-dimensional $\Fb_p$ vector space.
Letting $\G_{\ell,i}\subset \GQ$ be the $i$th ramification group at $\ell$ for a prime $\ell$, as defined in \cref{def:inert}, we can consider the fixed subspace of $V$ for each $\ell$ and $i$,
\[
V^{\ell,i} = \{v\in V : \rho(\sigma) v = v\ \forall \sigma \in \G_{\ell,i}\}.
\]
For each $\ell$ we can then define
\[
\nu_\ell(\rho) = \sum_{i = 0}^{\infty} \frac{1}{[\G_{\ell,0} : \G_{\ell,i}]} \dim\left(V/V^{\ell,i}\right),
\]
this quantity is in fact an integer \cite{}. %TODO
We then define our level by
\[
N(\rho) = \prod_{\substack{\ell \ne p\\ \ell\text{ prime}}} \ell^{\nu_\ell(\rho)},
\]
which is indeed a positive integer, and by construction it is coprime to $p$.

\begin{rmk}\label{rmk:unram}
Unwinding this definition when $\rho$ is unramified at some $\ell$, we see that each $V^{\ell,i}$ is in fact the whole of $V$, as the ramification groups involved are trivial. %TODO check
Hence in this case $\nu_\ell(\rho) = 0$ and so $N(\rho)$ is only divisible by the primes $\ell \ne p$ at which $\rho$ is ramified.
\end{rmk}

The definition above is that of the \emph{Artin conductor} of a representation, but with the $p$ part ignored. %TODO

\cref{thm:level} was that when $\rho$ comes from a modular form $f$ this quantity divides the weight of $f$.
With that in mind conjecturing that the optimal weight is $N(\rho)$ is fairly logical.


\subsection{The character and the weight mod $p-1$}
Beginning with a Galois representation of Serre type as before, we now define a character
\[
\varepsilon(\rho)\colon  (\ZZ/N(\rho)\ZZ)^* \to \Fb_p^*.
\]
We also state the class of $k(\rho)$~mod~$p-1$, though the full definition of $k(\rho)$ will be given in the next section.

Given a continuous mod $p$ Galois representation $\rho$ we can compose with the determinant map to obtain a continuous character
\[
\det \rho\colon \GQ \to \Fb_p^*.
\]
As outlined in \cref{rmk:ctsfin} the image of a continuous mod $p$ Galois representation is finite.
Hence the image of $\det \rho$ is a finite multiplicative subgroup of a field, so the image is cyclic.

We now compute the conductor of $\det\rho$.
Let $V_1$ be the 2-dimensional vector space realising $\rho$ and $V_2$ be the 1-dimensional vector space realising $\det\rho$.
If $\det\rho|_{G_{\ell, i}}$ is not trivial then $\rho|_{G_{\ell,i}}$ cannot be trivial, hence $\dim(V_2/V_2^{\ell,i}) > 0$ implies $\dim(V_1/V_1^{\ell,i}) > 0$.
As $0 \le \dim(V_2/V_2^{\ell,i}) \le 1$ we get that
\[
\dim(V_2/V_2^{\ell,i})\le \dim(V_1/V_1^{\ell,i})
\]
for all $\ell$ and $i$, and hence
\[
\nu_\ell(\det\rho) \le \nu_\ell(\rho).
\]
This gives us that
\[
N(\det\rho) \mid N(\rho).
\]

% TODO do we want artin cond here

We can therefore identify $\det\rho$ with a homomorphism
\[
(\ZZ/pN(\rho)\ZZ)^* \to \Fb_p^*,
\]
or equivalently with a pair of homomorphisms
\begin{align*}
\phi\colon& (\ZZ/p\ZZ)^* \to \Fb_p^*,\\
\varepsilon\colon& (\ZZ/N(\rho)\ZZ)^* \to \Fb_p^*.
\end{align*}

The $\varepsilon$ we get by doing this is our character $\varepsilon(\rho)$.

The group $(\ZZ/p\ZZ)^*$ is cyclic of order $p-1$ so any generator must map to an element of order dividing $p-1$ in $\Fb_p^*$, hence its image must be in the prime field.
Therefore this map is of the form
\[
x \mapsto x^h,
\]
for some $h \in \ZZ/(p-1)\ZZ$.
So $\phi = \chi_p^h$, where $\chi_p$ is the mod $p$ cyclotomic character. %TODO p-adic or mod p?

So we have written
\[
\det\rho = \varepsilon \chi_p^h
\]
and so comparing with \cref{rmk:detrho} the of $h \mod{p-1}$ had better be the same as $k(\rho)-1\mod{p-1}$.
Now it remains to define the actual value of $k(\rho)$.



\subsection{The optimal weight}
We now come to the final ingredient in Serre's recipe, that of the weight $k(\rho)$.
Given our Galois representation
\[
\rho \colon \GQ \to \Aut(V)
\]
recall from \cref{sec:gals} that we can form a representation of $\Gp$ by composing with a restriction map $\Gp \to\GQ$, to obtain
\[
\rho_p\colon \Gp \to \GQ \to \Aut(V).
\]
The definition of $k(\rho)$ will only depend on this $\rho_p$ (in fact only on $\rho_p|_{I_p}$) and therefore the weight will only reflect the behaviour at $p$ of the representation. %TODO is this what serre writes?
We will from here on refer to $k(\rho)$ as $k(\rho_p)$ to emphasise this.

\begin{defn}\label{def:semisimp}
The \emph{semisimplification} of a 2-dimensional representation $\rho$ is another representation $\rss$ obtained as follows.
If $\rho$ is irreducible (and hence semisimple) we leave it as it is and set $\rss = \rho$.
Otherwise after picking a particular basis $\rho$ is given by
\[
\begin{pmatrix}
\phi_1 & * \\
0      & \phi_2
\end{pmatrix}.
\]
The semisimplification $\rss$ is then the representation given by
\[
\begin{pmatrix}
\phi_1 & 0 \\
0      & \phi_2
\end{pmatrix}.
\]
Which is indeed semisimple, as you would hope.

In the general case the process of semisimplification is analogous, and is obtained by taking the direct sum of the Jordan--H\"older constituents of a representation, though for us the above description suffices.
\end{defn}

Now $\rss|_{I_{p,w}}$ is trivial and therefore $\rss$ may be viewed as a representation of $I_{p,t}$.
This representation is diagonalisable and so it is given by a pair of characters
\[
\phi_1,\,\phi_2\colon I_{p,t} \to \Fb_p^*.
\]

\begin{prop}
Both of the characters $\phi_1$ and $\phi_2$ are of the same level, and that level is either 1 or 2.

Moreover if they are both of level 2 then they are $p$th powers of each other.
\end{prop}
\begin{proof}

%TODO set \{\phi_1, \phi_2\} is stable under $p$th powering.
We have two possibilities, either taking the $p$th power fixes both $\phi_1$ and $\phi_2$ or it swaps them.
If they are both fixed then they must be of level 1.
Otherwise, if they swap under $p$th powering, each of them is fixed under powering by $p^2$ and hence they are of level 2.
\end{proof}

We now treat three different cases separately, based on the levels of the characters just obtained and whether $\rho|_{I_{p,w}}$ is trivial.


\subsubsection{The level 2 case}\label{sec:l2}
When the characters are of level 2 we can write them in terms the fundamental characters $\psi_1$ and $\psi_2$ of level 2 (as defined in \cref{def:fund}) and use this description to define $k(\rho_p)$.
Specifically we can write $\phi_1$ as
\[
\phi_1 = \psi_1^a\psi_2^b
\]
with $0\le a,\,b\le p-1$.
If $a = b$ then $\phi_1 = (\psi_1 \psi_2)^a$, which contradicts $\phi_1$ being of level 2 as $\psi_1\psi_2$ is a level 1 character (see \cref{rmk:prodchar}).
Now we observe that
\[
\phi_2 = \phi_1^p = (\psi_1^a\psi_2^b)^p = \psi_2^a\psi_1^b,
\]
and so by switching the places of $\phi_1$ and $\phi_2$ we may assume that in fact $0\le a < b\le p-1$.

This now looks a bit like the supersingular case of \cref{thm:super}.
So we massage our representation into the form seen in the theorem by factoring some characters out of $\rss$ to get
\[
\begin{pmatrix}
\psi_1^b\psi_2^a & 0 \\
0                & \psi_1^a\psi_2^b
\end{pmatrix} =
\psi_2^a\psi_1^a\begin{pmatrix}
\psi_1^{b-a} & 0 \\
0            & \psi_2^{b-a}
\end{pmatrix} =
\chi_p^a\begin{pmatrix}
\psi_1^{b-a} & 0 \\
0            & \psi_2^{b-a}
\end{pmatrix}.
\]
If we were just considering the matrix on the right we would like to set $k(\rho_p) - 1 = b - a$, however we have twisted by $\chi_p^a$.
Recalling \cref{prop:theta} we make the definition
\[
k(\rho_p) - 1 = b - a + a(p+1),
\]
or equivalently
\begin{equation}\label{eq:l2}
k(\rho_p) = 1 + pa + b.
\end{equation}
%TODO range

As we have $0 \le a< b \le p-1$ we see that
\[
2\le k(\rho_p) \le 1 + p(p-2) + p-1 = p(p-1).
\]


\subsubsection{The level 1 tame case}\label{sec:l1t}
Assuming $\phi_1$ and $\phi_2$ are of level 1 and the action of $I_p$ on $V$ is semisimple we can write
\[
\rho_p |_{I_p} = \begin{pmatrix}
\phi_1 & 0 \\
0      & \phi_2 \end{pmatrix} = \begin{pmatrix}
\chi_p^b & 0 \\
0      & \chi_p^a \end{pmatrix}.
\]
So we obtain integers $a$ and $b$ defined modulo $p-1$, we can assume that $0\le a \le b \le p-2$ by switching $\phi_1$ and $\phi_2$ if necessary.
This looks a little like the ordinary case of \cref{thm:ordinary} so we factor out a character again to get something that looks exactly like the theorem,
\[
\begin{pmatrix}
\chi_p^b & 0 \\
0        & \chi_p^a \end{pmatrix}=
\chi_p^a\begin{pmatrix}
\chi_p^{b-a} & 0 \\
0            & 1 \end{pmatrix}.
\]
If we had only the right hand matrix we would wish to set $k(\rho_p) - 1 = b-a$, but once again we have a twist.
Taking this into account we try and set
\[
k(\rho_p) - 1 = b - a + a(p+1),
\]
or equivalently
\[
k(\rho_p) = 1 + pa + b
\]
as above, but there is a small issue this time.
We could have $a = b = 0$ and this definition would give us $k(\rho_p) = 1$, but using the above definition there are no weight 1 modular forms.
So we add $p-1$ in this case, in the hope that we obtain a congruent form that actually exists %TODO
\begin{equation}\label{eq:l1t}
k(\rho_p) = \begin{cases}
1 + pa + b & \text{if }(a,\,b) \ne (0,\,0), \\
         p & \text{if }(a,\,b) = (0,\,0).
\end{cases}
\end{equation}

We now get
\[
2\le k(\rho_p) \le 1 + p(p-2) + p-2 = p^2 - p - 1
\]
unless $p = 2$, where the above inequality makes no sense, in which case we necessarily have $k(\rho_p) = 2$.


\subsubsection{The level 1 non-tame case}\label{sec:l1nt}
The final case is where $\phi_1$ and $\phi_2$ are of level 1 but the action of $I_{p,w}$ on $V$ is not trivial.
If we consider the subspace of $V$ fixed by $I_{p,w}$

\[
\rho_p = \begin{pmatrix}
\theta_2 & * \\
0        & \theta_1 \end{pmatrix},
\]
we may decompose $\theta_1$ and $\theta_2$ as $\chi_p^\beta \varepsilon_1$ and $\chi_p^\alpha \varepsilon_2$ respectively, where $\varepsilon_1$ and $\varepsilon_2$ are unramified characters and $\alpha,\,\beta \in \ZZ/(p-1)\ZZ$.
Using this decomposition we may write the restriction to $I_p$ as
\[
\rho_p|_{I_p} = \begin{pmatrix}
\chi_p^\beta & * \\
0          & \chi_p^\alpha \end{pmatrix}.
\]
We fix representatives $\alpha$ and $\beta$ now such that
\begin{align*}
0&\le \alpha \le p - 2,\\
1&\le \beta \le p - 1.
\end{align*}
Then we set
\begin{align*}
a &= \min(\alpha, \beta),\\
b &= \max(\alpha, \beta).
\end{align*}

If $\beta \ne \alpha + 1$ we let
\begin{equation}\label{eq:l1nt}
k(\rho_p) = 1 + pa + b,
\end{equation}
as we did in \cref{sec:l2}.

If we instead have $\beta = \alpha + 1$ we have to distinguish two cases, which, after Serre, we shall refer to as \emph{peu ramif\'e} and \emph{tres ramif\'e}.
In order to define these cases we need to study the field $K$ cut out by the kernel of $\rho_p$.

If we are in the peu ramif\'e case we define $k(\rho_p)$ as before via
\begin{equation}\label{eq:peu}
k(\rho_p) = 1 + pa + b = 2 + \alpha(p+1),
\end{equation}
the second equality holds as we have $\beta = \alpha + 1$.

Finally if we are in the tres ramif\'e case we add $p-1$ to \cref{eq:peu}, unless $p=2$, in which case we add 2 instead, obtaining
\begin{equation}\label{eq:tres}
k(\rho_p) = \begin{cases}
1 + pa + b + p - 1 = (\alpha + 1)(p+1) & \text{ if }p\ne 2,\\
4 & \text{ if } p = 2.
\end{cases}
\end{equation}


\subsection{A small issue}
It needs mentioning that the conjecture exactly as stated above is in fact \emph{incorrect} and we have the following counterexample (due to Serre).

\begin{ex}
Let . %TODO
\end{ex}

This problem is fairly isolated and only arises from trying to specify the character in addition to the weight and level when we work with mod~2 or mod~3 Galois representations.
The reason this happens is due to the failure of


\subsection{The proof}
As mentioned at the start, this conjecture is in fact now a theorem.
We will not go into detail about the proof here and restrict ourselves to the following remarks:

It was known that the qualitative and refined forms were equivalent before either was known in general.
The reduction of the refined form to the qualitative is due to a large number of people and .
Once this reduction has been obtained it is only necessary to .


% ************************************************************************
% Examples
% ************************************************************************
\section{Examples}
One of the great things about Serre's conjecture, even if it were not yet known to be correct, is the fact that it can be used in specific cases easily.
Specifically, given a Galois representation of Serre type we can calculate the optimal weight and level as detailed above, and then in many instances we can compute the associated space of modular forms and look for a form from which our Galois representation arises.

\begin{ex}
Let's return first to \cref{ex:delt} and check that everything we have just done is consistent with what we saw there.

So $\rho_{7}$ should have arisen from a normalised eigenform
\[
f\in S_{12}(1,\,\Id;\,\Fb_p) = \Fb_p\cdot \overline{\Delta},
\]
as indeed it did.
\end{ex}

Now we move to a new example arising from the Galois group of a number field.

\begin{ex}
Take the $K$ to be the splitting field of
\[
f = x^3 + ??x + ??.
\]
This has Galois group $S_3$ and we may consider the restriction map
\[
\rho\colon \Gal(\Qb/\QQ) \to \Gal(K/\QQ) \simeq S_3.
\]
We can turn this into a Galois representation using the fact that $S_3$ is isomorphic to $\GL_2(\FF_2)$ via the identification %TODO
\[
(1, 2) \mapsto \begin{pmatrix} 1 & 1 \\ 0 & 1 \end{pmatrix},\,
(1, 3) \mapsto \begin{pmatrix} -1 & 0 \\ 0 & 1 \end{pmatrix}.
\]
So $\rho$ can actually be viewed as a mod 2 Galois representation, what does Serre have to say about it?
Well for starters $N(\rho)$

So $\rho$ should come from some eigenform $f \in S_?(?,\,?;\,\Fb_2)$.
We can explicitly compute this space using, for example, Sage \cite{Sage}.
Doing this gives us that
\[
S_?(?,\,?;\,?) = \Fb_2 \cdot f.
\]

\end{ex}


% ************************************************************************
% Consequences
% ************************************************************************
\section{Consequences}
Serre's conjecture is a strong statement that implies many other difficult results within number theory.
We now mention briefly a few of these.
While parts of many of these results were obtained via other means long before Serre's conjecture was shown in general they still serve to demonstrate the power and usefulness of the conjecture.


\subsection{Unramified mod $p$ Galois representations for small $p$}
First let us examine a very direct consequence. %TODO non-existence
We take $\rho$ to be be a mod $p$ Galois representation for some $p < 11$ that is unramified outside of $p$.
In this case, due to the absence of ramification, $N(\rho)$ is simply 1 (recall \cref{rmk:unram}). %TODO weight
So Serre's conjecture predicts there is some mod $p$ cusp form of level 1 and weight ?? from which $\rho$ arises.
But there are no cusp forms of level 1 of weight $< 12$ and so such a representation cannot exist.

We can make similarly simple conclusions using this type of argument in other low dimensional cases too, for example if we 


\subsection{The Taniyama--Shimura--Weil conjecture}\label{sec:tan}


\subsection{The Artin conjecture}\label{sec:artin}
\begin{defn}
An \emph{Artin representation} is a complex Galois representation
\[
\rho\colon \GQ \to \GL_2(\CC).
\]
\end{defn}

We may consider the $L$-function
\[
L(s, \rho) = \prod_{p} L_p(s,\rho) = \prod_{p} \frac{1}{\det(I_n - p^{-s}\rho(\Frob_p) |_{V^{p,0}})}.
\]

Given any $L(s,\rho)$ we introduce a related function which has a nice functional equation.
We define
\[
\Lambda(s, \rho) = N^{s/2} (2\pi)^{-s} \Gamma(s)L(s,\rho),
\]
where $N$ is the Artin conductor, recalling the notation of \cref{subsec:level} this is given by
\[
N = \prod_{p}p^{\nu_p(\rho)}.
\]
This function satisfies
\[
\Lambda(1-s, \rho) = W(\rho)\Lambda(s,\rho),
\]
where $W(\rho)$ is a constant of absolute value 1, called the \emph{Artin root number}.

The following conjecture is a major question concerning this function.

\begin{conjecture}[Weak Artin conjecture]
Let
\[
\rho\colon \GQ \to \GL_n(\CC)
\]
be an Artin representation, then the meromorphic continuation of
\[
\Lambda(s,\rho)
\]
to the complex plane is holomorphic on the whole of $\CC$.
\end{conjecture}

In fact this follows from

\begin{conjecture}[Strong Artin conjecture]
Any Artin representation
\[
\rho\colon \GQ \to \GL_n(\CC),
\]
is modular, in the sense that it
\end{conjecture}

As the $L$-function of a ??

\begin{prop}
Serre's conjecture implies the strong Artin conjecture for odd 2-dimensional Artin representations.
\end{prop}
\begin{proof}
Given an odd Artin representation
\[
\rho\colon \GQ\to \GL_2(\CC)
\]

\end{proof}


\subsection{Modularity of abelian varieties}
The following result that can be deduced from Serre's conjecture does not obviously concern the objects related in the conjecture
%it does follow by the work of Ribet \cite{Ribet}.


\bibliographystyle{alpha}
\bibliography{essay}

\end{document}
